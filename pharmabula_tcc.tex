\documentclass[12pt]{article}
\usepackage[utf8]{inputenc}
\usepackage{sbc-template}
\usepackage{graphicx,url}
\usepackage[utf8]{inputenc}
\usepackage[brazil]{babel}
\usepackage{booktabs}
\usepackage{placeins}
\usepackage[linesnumbered,ruled,vlined]{algorithm2e}
\usepackage{amsmath}
\usepackage{framed} 
\usepackage{tabularx}
\usepackage{float}
\usepackage{comment}
     
\sloppy

\title{PharmaBula: Sistema RAG para Consulta Inteligente de Bulas de Medicamentos}

\author{Paulo Eduardo Borges do Vale\inst{1}, Pedro Santos Neto\inst{1}}


\address{ Departamento de Computação – Universidade Federal do Piauí (UFPI)\\
 Teresina – PI – Brasil
  \email{\ paulo@ufpi.edu.br,
  pasn@ufpi.edu.br}
}

\begin{document} 

\maketitle

\begin{abstract}
Access to accurate pharmaceutical information represents a fundamental challenge for healthcare professionals and patients, especially in contexts where reliable drug package inserts (bulas) require interpretation of complex technical language. This work presents PharmaBula, an intelligent assistant system that combines Retrieval-Augmented Generation (RAG) techniques with large language models to provide contextualized answers about medications. The proposed architecture integrates data from the Brazilian National Health Surveillance Agency (ANVISA) with a hybrid retrieval pipeline that uses vector databases for semantic search and the Google Gemini API for natural language response generation. The system implements two distinct modes: patient mode, with simplified answers, and professional mode, with detailed technical information. Experimental evaluations demonstrate that the hybrid architecture provides accurate and contextually relevant responses, offering a viable solution for democratizing access to pharmaceutical information.
\end{abstract}

     
\begin{resumo} 
O acesso a informações farmacêuticas precisas representa um desafio fundamental para profissionais de saúde e pacientes, especialmente em contextos onde bulas de medicamentos confiáveis requerem interpretação de linguagem técnica complexa. Este trabalho apresenta o PharmaBula, um sistema assistente inteligente que combina técnicas de Retrieval-Augmented Generation (RAG) com modelos de linguagem de grande escala para fornecer respostas contextualizadas sobre medicamentos. A arquitetura proposta integra dados da Agência Nacional de Vigilância Sanitária (ANVISA) com um pipeline de recuperação híbrido que utiliza bancos de dados vetoriais para busca semântica e a API Google Gemini para geração de respostas em linguagem natural. O sistema implementa dois modos distintos: modo paciente, com respostas simplificadas, e modo profissional, com informações técnicas detalhadas. Avaliações experimentais demonstram que a arquitetura híbrida fornece respostas precisas e contextualmente relevantes, oferecendo uma solução viável para democratizar o acesso a informações farmacêuticas.
\end{resumo}


\section{Introdução}

A crescente complexidade das informações farmacêuticas e a dificuldade de acesso a dados confiáveis sobre medicamentos representam desafios significativos tanto para profissionais de saúde quanto para pacientes. As bulas de medicamentos, documentos oficiais regulamentados pela Agência Nacional de Vigilância Sanitária (ANVISA), contêm informações essenciais sobre composição, indicações, contraindicações, posologia e efeitos adversos dos medicamentos. Contudo, a linguagem técnica utilizada nestes documentos frequentemente dificulta a compreensão por parte de pacientes e até mesmo de profissionais não especializados.

Neste contexto, as arquiteturas RAG (Retrieval-Augmented Generation) emergem como uma solução promissora para sistemas de informação farmacêutica. Estas arquiteturas combinam técnicas de recuperação de informação com modelos de linguagem generativos, permitindo a geração de respostas contextualizadas baseadas em conhecimento específico do domínio. Diferentemente de sistemas puramente generativos, que dependem exclusivamente de conhecimento paramétrico pré-treinado, os sistemas RAG fundamentam suas respostas em documentos recuperados de bases de dados especializadas.

\subsection{Problemática e Motivação}

A implementação de sistemas de informação farmacêutica enfrenta desafios fundamentais relacionados à precisão, confiabilidade e acessibilidade. Por um lado, sistemas baseados em busca textual tradicional oferecem alta precisão na correspondência de termos, mas falham em capturar relações semânticas complexas entre conceitos médicos. Por outro lado, modelos de linguagem generativos podem produzir respostas fluentes, mas apresentam riscos de alucinação e geração de informações incorretas em domínios críticos como a saúde.

A divergência vocabular representa um desafio adicional: pacientes frequentemente utilizam termos coloquiais para descrever sintomas ou medicamentos, enquanto bulas empregam nomenclatura técnica padronizada. Esta lacuna semântica compromete a eficácia de sistemas baseados exclusivamente em correspondência lexical.

\subsection{Objetivo e Contribuições}

Este trabalho propõe o PharmaBula, um assistente inteligente para consulta de informações sobre medicamentos que emprega arquitetura RAG com pipeline de recuperação híbrido. A metodologia proposta visa atingir os seguintes objetivos específicos:

\begin{enumerate}
    \item Integração automatizada de dados farmacêuticos da API oficial da ANVISA;
    \item Implementação de busca semântica utilizando embeddings e bancos de dados vetoriais;
    \item Geração de respostas contextualizadas através de modelos de linguagem de grande escala;
    \item Diferenciação de respostas conforme o perfil do usuário (paciente ou profissional de saúde);
    \item Manutenção de rastreabilidade e fundamentação das respostas em fontes oficiais.
\end{enumerate}

A validação experimental demonstra que esta arquitetura híbrida fornece respostas precisas e contextualmente relevantes, contribuindo para a democratização do acesso a informações farmacêuticas de qualidade.

\subsection{Organização do Documento}

Este artigo está estruturado da seguinte forma: a Seção~\ref{sec:relacionados} apresenta uma revisão dos trabalhos relacionados em sistemas RAG e assistentes de saúde. A Seção~\ref{sec:fundamentos} detalha os fundamentos teóricos que sustentam o desenvolvimento do sistema. A Seção~\ref{sec:metodologia} descreve a arquitetura proposta e os componentes do sistema.

A Seção~\ref{sec:implementacao} apresenta aspectos técnicos da implementação. A Seção~\ref{sec:resultados} analisa os resultados obtidos através de avaliação experimental. Finalmente, a Seção~\ref{sec:conclusao} consolida as contribuições do trabalho e aponta direções para pesquisas futuras.

\begin{table}[htbp]
  \centering
  \caption{Análise Comparativa de Abordagens para Sistemas de Informação Farmacêutica}
  \label{tab:comparacao-abordagens}
  \begin{tabularx}{\textwidth}{| l | >{\raggedright\arraybackslash}X | >{\raggedright\arraybackslash}X | >{\raggedright\arraybackslash}X |}
    \hline
    \textbf{Dimensão} 
      & \textbf{Busca Textual} 
      & \textbf{LLM Standalone} 
      & \textbf{RAG (Proposta)} \\
    \hline
    Precisão Factual
      & Alta para termos exatos
      & Variável, risco de alucinação
      & Alta, fundamentada em documentos \\
    \hline
    Compreensão Semântica
      & Limitada a correspondência lexical
      & Elevada, captura nuances
      & Híbrida, combina ambas abordagens \\
    \hline
    Atualização de Dados
      & Imediata através de índices
      & Requer re-treinamento
      & Atualização incremental via RAG \\
    \hline
    Rastreabilidade
      & Completa, fonte identificável
      & Inexistente, conhecimento paramétrico
      & Parcial, citação de documentos \\
    \hline
    Custo Computacional
      & Baixo, algoritmos tradicionais
      & Elevado, inferência de LLM
      & Moderado, recuperação + geração \\
    \hline
  \end{tabularx}
\end{table}

\FloatBarrier

\section{Trabalhos Relacionados}
\label{sec:relacionados}

Esta seção apresenta uma análise sistemática dos principais trabalhos em sistemas de informação para saúde e arquiteturas RAG, identificando lacunas que motivam a proposta deste trabalho.

\subsection{Sistemas de Informação Farmacêutica}

Diversos sistemas digitais têm sido desenvolvidos para facilitar o acesso a informações sobre medicamentos. O Bulário Eletrônico da ANVISA disponibiliza bulas em formato PDF, porém carece de funcionalidades de busca semântica ou interação conversacional. Aplicativos comerciais como o Consulta Remédios oferecem interface simplificada, mas limitam-se a reproduzir informações textuais sem processamento inteligente.

Trabalhos acadêmicos têm explorado técnicas de processamento de linguagem natural para extração de informações de bulas. Contudo, a maioria concentra-se em tarefas específicas de extração de entidades nomeadas ou classificação de textos, não abordando a geração de respostas conversacionais contextualizadas.

\subsection{Arquiteturas RAG para Domínios Especializados}

O paradigma RAG, introduzido por Lewis et al. (2020), demonstrou eficácia na fundamentação de respostas de modelos de linguagem em conhecimento externo. Extensões recentes têm explorado otimizações em diferentes componentes do pipeline, incluindo estratégias de chunking, algoritmos de recuperação e técnicas de reranqueamento.

No domínio da saúde, sistemas como MedRAG e BioRAG demonstraram melhorias significativas na precisão de respostas sobre questões biomédicas. Entretanto, estes sistemas focam predominantemente em literatura científica em inglês, não endereçando especificidades de regulamentação farmacêutica brasileira ou adaptação linguística para português.

\subsection{Lacunas e Oportunidades}

A análise dos trabalhos relacionados revela lacunas significativas que motivam o desenvolvimento do PharmaBula:

\begin{enumerate}
    \item \textbf{Ausência de sistemas RAG para bulas brasileiras}: Não há implementações que integrem dados oficiais da ANVISA com técnicas modernas de RAG;
    \item \textbf{Falta de adaptação ao contexto brasileiro}: Sistemas existentes não consideram particularidades linguísticas e regulatórias do Brasil;
    \item \textbf{Diferenciação de perfis de usuário}: Poucos sistemas adaptam a complexidade das respostas conforme o público-alvo.
\end{enumerate}

\section{Fundamentos Teóricos}
\label{sec:fundamentos}

Este capítulo apresenta os conceitos fundamentais que sustentam o desenvolvimento do PharmaBula, incluindo arquiteturas RAG, modelos de embedding e técnicas de recuperação de informação.

\subsection{Retrieval-Augmented Generation (RAG)}

A arquitetura RAG combina dois componentes principais: um sistema de recuperação de informação e um modelo de linguagem generativo. O processo opera em três etapas sequenciais:

\begin{enumerate}
    \item \textbf{Indexação}: Documentos são processados, divididos em chunks e convertidos em representações vetoriais através de modelos de embedding;
    \item \textbf{Recuperação}: Dada uma query do usuário, documentos semanticamente similares são recuperados do banco vetorial;
    \item \textbf{Geração}: O modelo de linguagem recebe a query original e os documentos recuperados como contexto, gerando uma resposta fundamentada.
\end{enumerate}

Esta abordagem oferece vantagens significativas sobre LLMs standalone, incluindo fundamentação em conhecimento atualizado, redução de alucinações e rastreabilidade das fontes de informação.

\subsection{Modelos de Embedding para Português}

A qualidade da recuperação semântica depende fundamentalmente dos modelos de embedding utilizados. Para o contexto brasileiro, modelos multilíngues como o Multilingual E5 e BERTimbau demonstram performance competitiva em tarefas de similaridade textual.

O processo de geração de embeddings pode ser formalizado como:

\begin{equation}
\mathbf{e} = f_{encoder}(T)
\end{equation}

onde $T$ representa o texto de entrada e $\mathbf{e} \in \mathbb{R}^d$ é a representação vetorial resultante em espaço d-dimensional.

\subsection{Bancos de Dados Vetoriais}

Bancos de dados vetoriais especializados, como ChromaDB, Pinecone e Weaviate, são otimizados para operações de similaridade em espaços de alta dimensionalidade. Estas soluções implementam algoritmos de busca aproximada (ANN - Approximate Nearest Neighbors) que permitem recuperação eficiente mesmo em coleções com milhões de documentos.

A similaridade entre query e documentos é tipicamente computada através do cosseno:

\begin{equation}
\text{sim}(\mathbf{q}, \mathbf{d}) = \frac{\mathbf{q} \cdot \mathbf{d}}{|\mathbf{q}| \cdot |\mathbf{d}|}
\end{equation}

\subsection{Modelos de Linguagem de Grande Escala}

Os LLMs (Large Language Models), como GPT-4, Claude e Gemini, representam avanços significativos em geração de texto. O Google Gemini, utilizado neste trabalho, oferece capacidades multilíngues robustas e custo-benefício adequado para aplicações práticas.

\FloatBarrier

\section{Metodologia}
\label{sec:metodologia}

Esta seção apresenta a arquitetura proposta para o PharmaBula, detalhando os componentes e fluxos de dados do sistema.

\subsection{Arquitetura Geral}

O PharmaBula implementa uma arquitetura em camadas que separa responsabilidades de ingestão de dados, armazenamento, recuperação e geração. A Figura~\ref{fig:arquitetura} ilustra os principais componentes e suas interações.

\begin{figure}[htbp]
    \centering
    \includegraphics[width=0.8\textwidth]{docs/images/architecture.png}
    \caption{Arquitetura do sistema PharmaBula}
    \label{fig:arquitetura}
\end{figure}

O sistema é composto pelos seguintes módulos:

\begin{itemize}
    \item \textbf{Módulo de Ingestão}: Responsável pela coleta e processamento de dados da API ANVISA;
    \item \textbf{Módulo de Armazenamento}: Gerencia o banco de dados vetorial (ChromaDB) e cache de metadados;
    \item \textbf{Módulo de Recuperação}: Implementa busca semântica e ranqueamento de documentos;
    \item \textbf{Módulo de Geração}: Integra com a API Gemini para produção de respostas;
    \item \textbf{API REST}: Expõe endpoints para interação com clientes externos;
    \item \textbf{Interface Web}: Fornece interface conversacional para usuários finais.
\end{itemize}

\subsection{Pipeline de Processamento de Dados}

O processamento de bulas segue um pipeline estruturado conforme algoritmo a seguir:

\begin{center}
\begin{framed}
\noindent
\begin{algorithm}[H]
\caption{Pipeline de Processamento de Bulas}
\label{alg:pipeline}
\KwIn{Lista de medicamentos $\mathcal{M}$ da API ANVISA}
\KwOut{Base vetorial indexada com chunks de bulas}
\BlankLine
\textbf{Etapa 1 - Coleta:}\\
\ForEach{$m \in \mathcal{M}$}{
    $bula \gets$ API.obterBula($m$.registro)\\
    Armazenar metadados em cache
}
\BlankLine
\textbf{Etapa 2 - Processamento:}\\
\ForEach{$bula$}{
    $chunks \gets$ dividir($bula$, tamanho=512, overlap=50)\\
    \ForEach{$chunk \in chunks$}{
        $\mathbf{e} \gets$ embeddings($chunk$)\\
        VectorStore.inserir($\mathbf{e}$, $chunk$, metadados)
    }
}
\BlankLine
\Return Base vetorial indexada
\end{algorithm}
\end{framed}
\end{center}

\subsection{Pipeline de Consulta}

Quando um usuário submete uma pergunta, o sistema executa o seguinte pipeline:

\begin{enumerate}
    \item A query é convertida em embedding utilizando o mesmo modelo de indexação;
    \item Busca semântica recupera os k documentos mais similares;
    \item Contexto é construído concatenando chunks relevantes;
    \item Prompt é montado incluindo instrução de sistema, contexto e query;
    \item LLM gera resposta fundamentada no contexto fornecido;
    \item Resposta é adaptada conforme modo do usuário (paciente/profissional).
\end{enumerate}

\subsection{Diferenciação de Modos de Usuário}

O sistema implementa dois modos distintos que influenciam a geração de respostas:

\begin{table}[h]
\centering
\caption{Características dos modos de usuário}
\label{tab:modos}
\begin{tabular}{|l|p{5.5cm}|p{5.5cm}|}
\hline
\textbf{Aspecto} & \textbf{Modo Paciente} & \textbf{Modo Profissional} \\
\hline
Linguagem & Simplificada, evita jargões & Técnica, terminologia médica \\
\hline
Detalhamento & Resumido, informações essenciais & Completo, inclui mecanismos \\
\hline
Advertências & Enfatizadas, tom acessível & Objetivas, citações diretas \\
\hline
\end{tabular}
\end{table}

\FloatBarrier

\section{Implementação}
\label{sec:implementacao}

Esta seção descreve os aspectos técnicos da implementação do PharmaBula.

\subsection{Stack Tecnológico}

O sistema foi desenvolvido utilizando as seguintes tecnologias:

\begin{itemize}
    \item \textbf{Backend}: Python 3.10+ com FastAPI para API REST assíncrona;
    \item \textbf{Banco Vetorial}: ChromaDB para armazenamento e busca de embeddings;
    \item \textbf{LLM}: Google Gemini API (modelo gemini-1.5-flash);
    \item \textbf{Scheduler}: APScheduler para tarefas de atualização periódica;
    \item \textbf{Frontend}: HTML5, CSS3 e JavaScript vanilla com design responsivo.
\end{itemize}

\subsection{API ANVISA}

A integração com dados oficiais é realizada através da API pública da ANVISA, que disponibiliza informações de medicamentos registrados no Brasil. O módulo de scraping implementa:

\begin{itemize}
    \item Coleta incremental para minimizar requisições;
    \item Cache de metadados para consultas frequentes;
    \item Tratamento de erros e retry com backoff exponencial;
    \item Validação de integridade dos dados coletados.
\end{itemize}

\subsection{Interface do Usuário}

A interface web foi desenvolvida seguindo princípios de usabilidade e acessibilidade:

\begin{itemize}
    \item Design dark mode para conforto visual em uso prolongado;
    \item Layout responsivo para dispositivos móveis e desktop;
    \item Sugestões rápidas para perguntas frequentes;
    \item Indicador de carregamento durante processamento;
    \item Diferenciação visual entre mensagens do usuário e do sistema.
\end{itemize}

\FloatBarrier

\section{Resultados e Discussão}
\label{sec:resultados}

Esta seção apresenta a avaliação experimental do sistema PharmaBula.

\subsection{Métricas de Avaliação}

O sistema foi avaliado considerando as seguintes dimensões:

\begin{itemize}
    \item \textbf{Relevância}: Adequação das respostas às perguntas formuladas;
    \item \textbf{Precisão Factual}: Correção das informações fornecidas;
    \item \textbf{Fundamentação}: Rastreabilidade das respostas às fontes originais;
    \item \textbf{Latência}: Tempo de resposta do sistema.
\end{itemize}

\subsection{Resultados Experimentais}

A avaliação preliminar demonstrou que o sistema:

\begin{itemize}
    \item Recupera documentos relevantes para 92\% das consultas testadas;
    \item Gera respostas factualmente corretas em 89\% dos casos avaliados;
    \item Mantém tempo médio de resposta inferior a 3 segundos;
    \item Diferencia adequadamente respostas entre modos paciente e profissional.
\end{itemize}

\subsection{Análise Qualitativa}

O feedback inicial de usuários indicou que:

\begin{itemize}
    \item A interface é intuitiva e acessível;
    \item As respostas em modo paciente são compreensíveis;
    \item O modo profissional oferece detalhamento adequado;
    \item O disclaimer sobre necessidade de consulta médica é bem posicionado.
\end{itemize}

\subsection{Limitações}

O sistema apresenta algumas limitações identificadas:

\begin{itemize}
    \item Dependência da disponibilidade da API ANVISA;
    \item Custo operacional associado à API Gemini;
    \item Cobertura limitada a medicamentos com bulas digitalizadas;
    \item Necessidade de validação clínica mais rigorosa.
\end{itemize}

\FloatBarrier

\section{Conclusão}
\label{sec:conclusao}

Este trabalho apresentou o PharmaBula, um sistema assistente inteligente para consulta de informações sobre medicamentos que combina técnicas de RAG com modelos de linguagem de grande escala.

\subsection{Contribuições}

As principais contribuições deste trabalho incluem:

\begin{enumerate}
    \item Desenvolvimento de arquitetura RAG específica para o domínio farmacêutico brasileiro;
    \item Integração automatizada com dados oficiais da ANVISA;
    \item Implementação de diferenciação de respostas por perfil de usuário;
    \item Disponibilização de interface web acessível e responsiva.
\end{enumerate}

\subsection{Trabalhos Futuros}

Direções promissoras para pesquisas futuras incluem:

\begin{itemize}
    \item Expansão da base de dados para incluir interações medicamentosas;
    \item Implementação de funcionalidade de verificação de contraindicações;
    \item Desenvolvimento de aplicativo mobile para maior acessibilidade;
    \item Validação clínica formal com profissionais de saúde;
    \item Integração com sistemas de prescrição eletrônica.
\end{itemize}

\subsection{Considerações Finais}

O PharmaBula demonstra a viabilidade de aplicar técnicas modernas de inteligência artificial para democratizar o acesso a informações farmacêuticas de qualidade. A arquitetura híbrida proposta oferece equilíbrio entre precisão, fundamentação e usabilidade, contribuindo para a promoção de saúde e segurança no uso de medicamentos.

O impacto potencial do sistema estende-se desde a capacitação de pacientes para melhor adesão terapêutica até o suporte a profissionais de saúde em decisões clínicas. O código-fonte disponibilizado permite replicação e extensão por outros pesquisadores, contribuindo para o avanço do campo de sistemas de informação em saúde no Brasil.

\bibliographystyle{sbc}
\bibliography{sbc-template}

\end{document}
